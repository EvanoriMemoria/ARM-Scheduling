\section{Evolution of Asymmetric Scheduling 2015}
\subsection{Previous Scheduler (Before 2015)}
Linux previously had three parts that all worked side-by-side in order to try to improve energy efficiency without compromising throughput.\\
The three parts are as follows:
\begin{enumerate} 
    \item Linux scheduler (Completely Fair Scheduler - CFS)
    \item Linux cpuidle
    \item Linux cpufreq
\end{enumerate}

\subsubsection{CFS}
The Linux Completely Fair Scheduler is very good at optimizing throughput by throwing processes onto open cores. However, by default it does not have any knowledge of energy usage or efficiency. When implementing Energy Aware Scheduling, the CFS is given the ability to access and read the power usage values of every process, as well as previous power usage, in addition to every core's idling statistics.\cite{EAS2015}

\subsubsection{cpu-idle}
The Linux cpuidle system attempts to use heuristics to save energy by putting the processor into an idle state without affecting performance. The cpuidle system has absolutely no control over the processes that are put onto specific cores or when, and is only capable of reacting to what the CFS throws at it.\cite{EAS2015}

\subsubsection{cpu-freq}
The cpufreq (cpu frequency) subsystem uses heuristics to throttle up or down the frequency of a core depending on the load in order to save on power - again without affecting performance. This system is also entirely separate from the CFS \textit{and} cpuidle. \cite{EAS2015}

\subsubsection{Previous Scheduler drawbacks}
The first issue we see with this three-part system is that the bottom two systems (cpuidle and cpufreq) are both reactive to what the CFS does, and since the CFS is entirely unaware of power management it frequently breaks their predictions and results in both power efficiency loss \textit{and} speed loss. In addition, the fact that they are reactive means they have to be run frequently themselves in order to keep up with what is happening in the processor in real time which creates more overhead management. \cite{EAS2015}

\subsection{Current Scheduler (After 2015)}
The current asymmetric multicore processor scheduler implementation is known as Energy Aware Scheduler or EAS. As the name implies, it takes into account the energy usage of each process and allocates the process to a core for the duration of the process. It should also be noted that EAS for Linux is being developed jointly by Arm and Linaro, it still has much room for improvement, specifically at full load.\cite{EASp2015}

\subsubsection{Energy Aware CFS}
Under EAS the Completely Fair Scheduler has undergone many changes, it still performs the same function and prioritizes throughput, but now takes into consideration both idle state and frequency. Effectively both cpuidle and cpufreq were reworked and their abilities given to the CFS.\cite{EASp2015}

\subsubsection{sched-idle}
The new system makes the scheduler aware of the idle states of every core and cluster. It prioritizes placing new processes onto the core with the shallowest idle state - that is to say - the one using the most power while idle. In so doing we lose less power savings than waking up one of the other cores which is more deeply in the idle state, and it responds more quickly as well. Therefore being both faster and more energy efficient. \cite{EASp2015}

\subsubsection{sched-freq}
Mainline Linux kernel is capable of storing information about processes and their load times. CPU Operating Performance Points then change almost immediately to accommodate the new process. Whereas before, cpufreq's main downfall was that since it was a run process it would have to wait until it's turn came to be run. Due to this it would often tune the frequency too late, making the process suffer under lower frequency (speed loss) or have the core at too high of a frequency (energy loss). The results were significant inefficiencies.\cite{EASp2015}

\subsection{Impacts}
It is clear that taking energy into account when scheduling is a vast improvement over blind scheduling with heuristics thrown on top, however, there are still issues that can arise from this scheduling process. Namely, the EAS works very well when the cpu load is interspersed, but when the load is frequently capped, and all cores are running at full capacity, I expect the scheduler will be unable to act on its priorities and begin acting as a completely fair scheduler once again. This does have limited impact on most user-side machines, however, there are certainly cases where having an EAS for full capacity would be beneficial. It should be noted that in order to do balance both speed and power consumption, the scheduler may have to hold process back from running for a time, meaning it would no longer be a Completely Fair Scheduler.